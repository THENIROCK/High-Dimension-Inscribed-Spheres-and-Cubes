Geometric reasoning can aid in developing a better visual intuition for higher dimensional phenomena. This chapter aims to reason for why the IK-Sphere radius diverges in terms of space consumption (volume) of the IK-Sphere, the unit $n$-spheres it `kisses', and its bounding $n$-cube on which the $n$-spheres are situated.

\begin{definition}[Volume]
    Here, volume has been generalised across $n$ dimensions as the number of unit $n$-cubes of space a shape occupies. For example, `length' is volume in one-dimensional space, `area' is volume in two-dimensional space, as is `volume' in three-dimensional space. 
\end{definition}

\section{The Volume of the n-Sphere}
The volume of an $n$-sphere is given as \cite{formula_n_sphere}:
\begin{equation}\label{eq:vol of sphere}
    V_n(r)=\frac{\pi^{\frac{n}{2}}}{\left(\frac{n}{2}\right)!}r^n
\end{equation}

\subsection{Spherical Coordinates}
The use of spherical coordinates can greatly simplify the derivation of the formula for the volume of an $n$-sphere. Consider ... using cartesian coordinates is very difficult, However, with spherical coordinates, one can ...

\subsection{Derivation of the Formula for an $n$-Sphere's Volume}

\section{Implications of the Higher Dimensional Unit Sphere on the IK-Sphere}
First, we must determine how the space consumption of the unit $n$-sphere changes as dimensionality $n$ increases. The reasoning here uses the result that as $n \to \infty$, a unit $n$-sphere's volume will tend to zero \cite{athreya2008unit}, resulting in the volume (and hence, the radius) of the IK-Sphere, which expands to fill the space between the surrounding unit $n$-spheres, expanding to infinity. 

\begin{lemma}\label{lemma:unit sphere volume tends to zero}
    As dimensionality increases a unit sphere's volume converges to zero.
\end{lemma}
\begin{proof}
    First, notice that $r_n$ will always equal 1 because it is a unit sphere and that $r^n=1^n$ which always equals 1 for all $n \in \mathbb{N}$. Thus, we can simplify the expression (Equation \ref{eq:vol of sphere}) on the right hand side to
    \begin{equation*}
        V_n=\frac{\pi^{\frac{n}{2}}}{\left(\frac{n}{2}\right)!}.
    \end{equation*}
    Next, the limit of the right hand side can be taken as $n$ approaches infinity:
    \begin{equation*}
        \lim_{n\to\infty}\frac{\pi^{\frac{n}{2}}}{\left(\frac{n}{2}\right)!}.
    \end{equation*}
    To evaluate this limit we can use the squeeze theorem whereby if the factors of an expression tend to a limit, then that expression itself must also tend to the same limit. Letting $m=\frac{n}{2}$ allows for a simpler expansion of the expression as a series to
    \begin{align*}
        \frac{\pi ^ m}{m!} &= \left( \frac{\pi}{m} \times \frac{\pi}{m-1} \times ... \times \frac{\pi}{1}\right)\\
        &= \left( \frac{\pi}{1} \times ... \times \frac{\pi}{m-1} \times \frac{\pi}{m} \right).
    \end{align*}
    As can be seen with $\pi$ as a constant on the numerator, the fractions are being multiplied by increasingly smaller fractions in the series ($... \times \frac{\pi}{m-2} \times \frac{\pi}{m-1} \times \frac{\pi}{m}$). As a result, as $m$ increases, this series will approach 0. This implies that as $\frac{n}{2}$ increases (because $m=\frac{n}{2}$), i.e. as $n$ increases,
    \begin{align*}
         \lim_{n\to\infty}\frac{\pi^{\frac{n}{2}}}{\left(\frac{n}{2}\right)!}=0
    \end{align*}
    as required.
\end{proof}

\begin{figure}[H]
    \centering
    % \begin{tikzpicture}
    %     \begin{axis}[
    %         axis lines = left,
    %         xlabel = \(n\),
    %         ylabel = {\(V_n\)},
    %     ]
    %     %Below the red parabola is defined
    %     \addplot [
    %         domain=0:20, 
    %         samples=100, 
    %         color=red,
    %     ]
    %     {(pi^(x/2))/(Gamma(x/2))};
    %     \addlegendentry{\(\sqrt{n}-1\)}
    %     %Here the blue parabola is defined
    %     \end{axis}
    % \end{tikzpicture}
    \begin{tikzpicture}[]
        \begin{axis}[
        xmin = 0, xmax = 20, 
        %ymin = -3.5, ymax = 3.5,  
        restrict y to domain=0:20,
        axis lines = left,
        axis line style={-latex},  
        xlabel={$n$}, 
        ylabel={$V_n$},
        %enlarge x limits={upper={val=0.2}},
        enlarge y limits=0.05,
        % x label style={at={(ticklabel* cs:1.00)}, inner sep=5pt, anchor=north},
        % y label style={at={(ticklabel* cs:1.00)}, inner sep=2pt, anchor=south east},
        ]
        
        \addplot[color=red, samples=222, smooth, 
        domain = 0:20] gnuplot{pi^(x/2)/gamma((x/2)+1)};
        \foreach[evaluate={\N=\n-1}] \n in {0,...,20}{%
            \addplot [domain=0:6, samples=2, densely dashed, thin] (\N, x);
        }%
        
        \addlegendentry[align=left]{\\$V_n=\frac{\pi^{\frac{n}{2}}}{\left(\frac{n}{2}\right)!}$\\}
        % \foreach[evaluate={\N=\n-1}] \n in {0,...,-5}{%
        % \addplot[color=red, samples=555, smooth,  
        % domain = \n:\N] gnuplot{gamma(x)};
        % %
        % \addplot [domain=0:6, samples=2, densely dashed, thin] (\N, x);
        % }%
        \end{axis}
    \end{tikzpicture}
    \caption{Volume of unit $n$-sphere tends to zero as dimension $n$ decreases.}
    \label{fig:unit sphere volume graph}
\end{figure}

Graphically, the limit of $V_n$ as $n \to 0$ can be observed (see Figure \ref{fig:unit sphere volume graph}). Due to the radius of the IK-Sphere expanding to `kiss' the unit spheres centered on the vertices of the bounding $n$-cube, if the volume of the unit $n$-spheres converges to zero as dimensionality increases, that would explain why the radius reaches out to infinity to `kiss' them -- it never can. If that is the case, we would expect the volume of the IK-Sphere to diverge alongside its radius as it expands to touch the unit $n$-spheres surrounding it. However, it may also be noticed that until $n=5$, the volume of the unit $n$-sphere increases. This would imply that the volume of the IK-Sphere should decrease for $n\leq 5$. However, ultimately, the volume should tend to infinity.

\begin{theorem}\label{theorem:volume of IK-Sphere}
As the number of dimensions $n$ increases, the volume of an IK-Sphere will expand to match the decreasing volume of its surrounding unit $n$-spheres. 
\end{theorem}

\begin{proof}
    We are required to evaluate the limit as $r_n$ and $n$ approach infinity of the volume of the IK-Sphere. Expressed in limit form, this is
    \begin{align*}
        \lim_{(r,n)\to\infty}V_n(r)&=\lim_{(r,n)\to\infty}\frac{\pi^{\frac{n}{2}}}{\left(\frac{n}{2}\right)!}r^n
    \end{align*}
    In this case, $r_n$ is dependent on $n$ and can be written as $r_n=\sqrt{n}-1$ (see Equation \ref{radius of ik sphere}) as we derived in Chapter 1. Therefore, the limit of two variables can be converted to a limit in one variable and becomes easier to solve. Substituting $r_n=\sqrt{n}-1$, the right hand side can be rewritten as
    \begin{align*}
        \lim_{n\to\infty}\frac{\pi^{\frac{n}{2}}{\left(\sqrt{n}-1\right)}^n}{\left(\frac{n}{2}\right)!}.
    \end{align*}
    For an arbitrary function $f(x)$, if $$\left(\lim_{x \to \infty} f(x)\right)^{-1}=\lim_{x \to \infty} f(x)^{-1}$$ is true, then by proving that $(V_n(r))^{-1} \to 0$ as $n \to \infty$, we can prove the inverse: $V_n(r) \to \infty$ as $n \to \infty$.
    
    \noindent
    We will now find the limit as $n \to \infty$ of $(V_n(r))^{-1}$ 
    \begin{align*}
        \lim_{n\to\infty}(V_n(r))^{-1}=\lim_{n\to\infty}\frac{\left(\frac{n}{2}\right)!}{\pi^{\frac{n}{2}}{\left(\sqrt{n}-1\right)}^n}.
    \end{align*}
    \noindent
    Let $m=\frac{n}{2}$ and expand $V_n(r)$ as a series
    \begin{align*}
        \frac{\pi^m{\left(\sqrt{2m}-1\right)}^{2m}}{m!}=\frac{m}{\pi(\sqrt{2m}-1)^2} \times \frac{m-1}{\pi(\sqrt{2m}-1)^2} \times ... \times \frac{1}{\pi(\sqrt{2m}-1)^2}
    \end{align*}
\end{proof}

\begin{figure}[H]
    \centering
    % \begin{tikzpicture}
    %     \begin{axis}[
    %         axis lines = left,
    %         xlabel = \(n\),
    %         ylabel = {\(V_n\)},
    %     ]
    %     %Below the red parabola is defined
    %     \addplot [
    %         domain=0:20, 
    %         samples=100, 
    %         color=red,
    %     ]
    %     {(pi^(x/2))/(Gamma(x/2))};
    %     \addlegendentry{\(\sqrt{n}-1\)}
    %     %Here the blue parabola is defined
    %     \end{axis}
    % \end{tikzpicture}
    \begin{tikzpicture}[]
        \begin{axis}[
        xmin = 0, xmax = 20, 
        %ymin = -3.5, ymax = 3.5,  
        restrict y to domain=0:20,
        axis lines = left,
        axis line style={-latex},  
        xlabel={$n$}, 
        ylabel={$V_n$},
        %enlarge x limits={upper={val=0.2}},
        enlarge y limits=0.05,
        % x label style={at={(ticklabel* cs:1.00)}, inner sep=5pt, anchor=north},
        % y label style={at={(ticklabel* cs:1.00)}, inner sep=2pt, anchor=south east},
        ]
        
        \addplot[color=red, samples=222, smooth, 
        domain = 0:20] gnuplot{pi^(x/2)/gamma((x/2)+1)};
        \foreach[evaluate={\N=\n-1}] \n in {0,...,20}{%
            \addplot [domain=0:6, samples=2, densely dashed, thin] (\N, x);
        }%
        
        \addlegendentry[align=left]{\\$V_n=\frac{\pi^{\frac{n}{2}}}{\left(\frac{n}{2}\right)!}$\\}
        % \foreach[evaluate={\N=\n-1}] \n in {0,...,-5}{%
        % \addplot[color=red, samples=555, smooth,  
        % domain = \n:\N] gnuplot{gamma(x)};
        % %
        % \addplot [domain=0:6, samples=2, densely dashed, thin] (\N, x);
        % }%
        \end{axis}
    \end{tikzpicture}
    \caption{Volume of IK-sphere tends to infinity as dimension $n$ decreases.}
    \label{fig:IK sphere volume to infinity graph}
\end{figure}

Graphically, there is no decrease in volume for $n \leq 5$ as was expected. Hence, there must be another factor contributing to the constant increase in volume. It may be noticed, however, that the volume increase after $n=5$ becomes much steeper. This implies that there is some effect from the surrounding unit sphere volumes increasing before $n=5$. We may now realise that the IK-Sphere volume also depends on the volume of its bounding $n$-cube on whose vertices the unit spheres are situated on. 

The volume of an $n$-cube with side length 2 units is given by $V_n = 2^n$.

We may graph the volumes of the IK-Sphere, unit $n$-sphere and bounding $n$-cube to reason how the IK-Sphere expands to fill the space between its surrounding $n$-spheres.
