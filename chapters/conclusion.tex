\section{Conclusion}
In conclusion, to answer the research question:
\researchquestion{}

There were two parts to the question. Firstly, we answer why the radius of an IK-Sphere tends to infinity as dimensionality increases:
The algebraic reason would be that when the limit of the expression for the radius $r$ of an IK-Sphere in terms of dimension $n$ is evaluated as $n$ approaches infinity, the radius tends to infinity as follows (see Theorem \ref{theorem:radius of IK-Sphere}):
\begin{align*}
    r = \sqrt{n}-1\\
    \lim_{n \to \infty}\sqrt{n}-1 = \infty
\end{align*}

However, such abstraction is not very useful in understanding higher dimensions. As an aid to one's higher dimensional intuition, the geometric reason is that as the dimension increases, the space, determined by the volume of the unit $n$-spheres the IK-Sphere `kisses', between unit spheres becomes infinitely large because the unit $n$-spheres' volumes approach zero (see Lemma \ref{lemma:unit sphere volume tends to zero}). Additionally, the volume of the bounding $n$-cube on which those spheres are situated also tends to infinity. As a result, the radius of the IK-Sphere expands to fill the space and its volume also tends to infinity (see Theorem \ref{theorem:volume of IK-Sphere}) in an attempt to `kiss' its bounding unit spheres. 

The second part of the question was whether the IK-Sphere truly intersected with its `bounding' box in $n>4$ dimensions or if it was merely higher dimensional phenomena making it seem so. We found (see Chapter 2: Section \ref{section:verify intersection}) that the IK-Sphere in 4 dimensions intersected at the single coordinate $(1, 1, 1, 0)$, directly in the center of the face of its bounding cube's $w=0$ face, just as expected. Thus, we also conclude that there is truth to an IK-Sphere intersecting with its bounding $n$-cube as dimensionality increases past $n=4$.

\section{Implications and Further Investigation} 
The conclusion has implications relating to what is known as the curse of dimensionality for high dimensional data sets in machine learning \cite{raviv2020perm2vec}. We found that as you increase dimensions, that the amount of space between points increases somewhat exponentially (see Figure \ref{fig:IK sphere volume to infinity graph} and `Bounding $n$-Cube Volume' in Figure \ref{fig:All volumes comparison}). As a result, when data has too many variables, the machine learning algorithm has trouble pinpointing attributes and patterns because the data is so vastly spread out. One may investigate possible dimensional elimination techniques to mitigate this issue. Additionally, the conclusion has implications for error correcting codes. For example, when a CD or DVD has a scratch on it, changing the encoded data, an error correcting code is used to fill in and `repair' the missing data. If each coordinate in high dimensional space is a piece of data, the space is separated into as many spheres as possible and the data read from the disk is corrected to the coordinate of the center of the sphere it exists within. Thus, the closer together we can get the spheres, the more accurate data corrections we can achieve. However, we learned that the space between spheres can become increasingly larger as dimension increases which means that as dimensions increase, even more spheres can be packed around each other. Until now, this sphere packing problem has only been proven up to eight dimensions and in 24 dimensions \cite{klarreich2016sphere}. With our newfound intuition of higher dimensions, could we prove the remaining dimensions? It is unlikely. However, such an investigation could reveal more phenomena to add to our bank of higher dimensional intuition.