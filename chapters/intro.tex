\title{To what extent is it true that an inscribed sphere will expand outside of its bounding box as dimensionality increases?}
\author{Candidate Code: jmw310 - check}
\date{November 2021}

%\maketitle


\section{Introduction}
Our intuition was formed in three dimensions and is often misleading in higher dimensions. Edwin Abbott's Flatland ... Linear Algebra contains the necessary axioms for us to engage with higher dimensions. Here is an interesting source to cite \cite{Cipra_1993}.
\researchquestion{}
    We examine the patterns, counts, and behavior of coefficients of
powers of polynomials in a finite field. We use two methods to produce
matrices that let us quickly calculate the total number of occurences
of any coefficient, and examine some properties of the matrices. We
form conjectures on the maximum eigenvalues of these matrices and
methods for reducing their size. Furthermore, we examine the blocks
of coefficients that appear in the powers of polynomials when reduced
modulo some prime, and derive recursive formulae for the number of
blocks of varying sizes can appear for certain polynomials and primes. And here is a theorem.

\begin{theorem}
As the dimension $n$ tends to infinity, the radius $r_n$ of the inscribed sphere also tends to infinity.
\end{theorem}


\begin{proof}
The following equation is direct proof of this statement.
\begin{equation}
y=mx+c
\end{equation}
To prove it by contradiction try and assume that the statemenet is false,
proceed from there and at some point you will arrive to a contradiction. 
\end{proof}


\section{Conclusion}
In conclusion, to answer the research question, 
\researchquestion{}

\section{Further Investigation} 

