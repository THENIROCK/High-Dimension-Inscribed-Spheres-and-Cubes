%See extra_comments.tex for the comments that I've left. I'll number them in this main.tex file and I'll typeset them nicely in extra_comments.tex.
%I will number my comments by GABE1, GABE2, ... so that you can easily search for them.
%The final comment I have left is GABE[finalNumberWillUpdateWhenI'mDone]. Check comments GABE1 -> GABE[final...Done] for what I've left.
%I can't be bothered to typeset the comments I've already sent by email. You can put them into LaTeX yourself if you'd like to visualise them.


%preamble
\documentclass[a4paper, 12pt]{report}
%Paragraph jumps and indentation
\setlength{\parskip}{1.6em}
\setlength{\parindent}{1.25cm}
%Border
\usepackage[left=1in, right=1in, top=1in, bottom=1in]{geometry}
%Double spacing
\usepackage{setspace}
\doublespacing
%Packages
\usepackage{amsmath}
\usepackage{systeme}
\usepackage{amsthm}
\usepackage{graphicx}
\usepackage{pgfplots}
\pgfplotsset{width=10cm,compat=1.9}
\usepackage{amssymb}
\usepackage{epigraph}
%\usepackage[dvipsnames]{xcolor}
\usepackage{mathtools}
\usepackage{amsfonts}
\usepackage{titlesec}
\usepackage{nomencl}
\makenomenclature
    %% This code creates the groups
    % -----------------------------------------
    \usepackage{etoolbox}
    \renewcommand\nomgroup[1]{%
      \item[\bfseries
      \ifstrequal{#1}{L}{Linear Algebra}{%
      \ifstrequal{#1}{N}{Number sets}{%
      \ifstrequal{#1}{O}{Other symbols}{}}}%
    ]}
    % -----------------------------------------
\usepackage{biblatex}
\addbibresource{bibliography.bib}
%Images
\usepackage{graphicx}
\graphicspath{ {./imagces/} }
\usepackage{wrapfig}
\usepackage{float}
%Tables
\usepackage{multirow}
\usepackage{array}
\usepackage{tabu}
\titleformat{\section}
{\normalfont\large\bfseries}{\thesection}{1em}{}
\titleformat{\subsection}
{\normalfont\large\bfseries}{\thesubsection}{1em}{}
%Equation numbering
\counterwithin{equation}{section}
\usepackage{hyperref}
\urlstyle{same}

%theorems, lemmas, proofs
\newtheorem{theorem}{Theorem}[section]
\newtheorem{corollary}{Corollary}[theorem]
\newtheorem{lemma}[theorem]{Lemma}
\theoremstyle{definition}
\newtheorem{definition}{Definition}[section]
\renewcommand\qedsymbol{$\blacksquare$}
%https://tex.stackexchange.com/questions/223948/suggestions-for-nested-proofs
\newenvironment{subproof}[1][\proofname]{%
    \renewcommand{\qedsymbol}{$\square$}%
    \begin{proof}[#1]%
}{%
    \end{proof}%
}
\renewcommand{\listfigurename}{List of Figures\footnote{All illustrations and graphs in this essay were made by the author}}
  


\newcommand{\researchquestion}{\begin{quote}\begin{center}
        \textbf{RQ: Why  does  the  radius  of  an  IK-Sphere  tend  to  infinity  as  dimensionality increases and does it truly intersect with its bounding box in $n \geq 4$ dimensions?}\end{center}
\end{quote}}

\newenvironment{amatrix}[1]{%
  \left[\begin{array}{@{}*{#1}{c}|c@{}}
}{%
  \end{array}\right]
}

% We will externalize the figures
\usepgfplotslibrary{external}
\tikzexternalize

\begin{document}
    \section{GABE1 - DONE}
The phrase ``$\mathbb{R}^{3}$ embedded in $\mathbb{R}^{4}$" isn't technically accurate. You literally cannot embed $\mathbb{R}^{3} = \left\{ x \Hat{i} + y \Hat{j} + z \Hat{k} \mid x,y,z\in\mathbb{R} \right\}$ in $\mathbb{R}^{4} = \left\{ x \Hat{e_1} + y \Hat{e_2} + z \Hat{e_3} + w \Hat{e_4} \mid x,y,z,w\in\mathbb{R} \right\}$. I don't think that the examiners will be that particular about language but ``allowing $x$, $y$ and $z$ to vary through $\mathbb{R}$ while keeping $w$ fixed" is an alternative.

\section{GABE2 - DONE}
Later in this proof, you say that $\Vec{v}$ is in the plane $\Pi$ if $\Vec{v} \cdot \Vec{n} = 0$. This is, in fact, a defining condition for a plane, but you take it as something assumed. You should provide this information in Definition \ref{def:n-plane}.

\section{GABE3 - DONE}
In reference to ``[...] we can deduce that a face of the 4-cube will lie on one of the $x,y,z,w$ axes.", I think this is a poor way to phrase this. More accurate is to say that `a face of the 4-cube will have either $x$, $y$, $z$ or $w$ constant'. You can follow this with `Let us take $w$ constant, and choose the face which intersects with $(0,0,0,0)$, giving $w=0$.' which immediately leads to your ``[...] of the form (x,y,z,0) [...]" statement.

\section{GABE4}
Insufficient explanation of where equation \ref{eq:cube face} ($w = 0$) comes from. I'd like to see some working in the form:
\begin{align*}
    ax+by+cz+dw &= k \\
    \text{more steps of working go here} \\
    \therefore w &= 0
\end{align*}
At present, you kinda just have `we've got $w=0$ from $(x,y,z,0)$ and hence, that's the whole equation' without any regard for why $a=b=c=1$ and $k=0$.
\par You can obtain these values by substituting the points $(2,0,0,0)$, $(0,2,0,0)$, $(0,0,2,0)$ and $(2,2,0,0)$ (or whatever other 4 points lie in this plane) and solving linear equations. Should take under 30 words.

\section{GABE5, GABE6}
This isn't the defining equation of a 4-sphere. The defining equation is rather
\begin{align*}
    (x-X)^2 + (y-Y)^2 + (z-Z)^2 + (w-W)^2 = 1
\end{align*}
where $(X,Y,Z,W)$ is the centre of the 4-sphere in question. You can obtain this from the equation $||\Vec{r} - \Vec{c}|| = 1$ where $\Vec{c}$ is the position vector for the centre of the sphere and $\Vec{r} = \langle X, Y, Z, W \rangle$.
\par You can use this to introduce the equation for your IK-sphere (eq: \ref{eq:IK 4 sphere centered at 1}, where I've left the comment GABE6) more easily.

\section{GABE6}
In addition to GABE5, I've got another comment -- why are you supposing that the radius of the IK-sphere is 1? Should it not be $r_4$?

\section{GABE7}
Agh I simply can't stand the use of $\ni$ for ``such that". Using a colon (:) is far more standard notation, making your statement read ``$\forall \alpha : \alpha \geq 0$" as opposed to the current ``$\forall \alpha \ni \alpha \geq 0$".
\par If using $\ni$ for ``such that" is standard notation, go for it, but I'm really not sure that it is.

\section{GABE8}
Reformatting this statement as follows makes it read easier (and avoids using the symbol $0$ for both $0 \in \mathbb{R}$ and $0 = (0,0,0)$):
\begin{align*}
    \forall\alpha\in\mathbb{R}, \, (\alpha-1)^2 &\geq 0 \\
\end{align*}
And since $(x-1)^2 + (y-1)^2 + (z-1)^2 =0$, the only choice is that
\begin{align*}
    &\begin{cases}
    (x-1)^2 = 0 \\
    (y-1)^2 = 0 \\
    (z-1)^2 = 0 \\
    \end{cases} \\
    &\begin{cases}
    x=1\\
    y=1\\
    z=1\\
    \end{cases} \\
\end{align*}
(end of reformatting)

\section{GABE9}
See comment in chapter\textunderscore2.tex.

\section{GABE10}
The second sentence of this paragraph is grammatically incorrect or confusing to read to the point where I thought it was grammatically incorrect.

\section{GABE11}
The parenthesised text reading ``(extended factorial function for fractional and negative real values)" isn't totally accurate. `(extended factorial function for all real values except the non-positive integers)' is correct. I've lifted this text from Wikipedia, so rephrase it before making this change.
\par Also, you should say in the  equation immediately following this that $\left(\frac{3}{2}\right)! = \Gamma\left(\frac{3}{2} + 1 \right)$ to be explicit. It doesn't cost any words, so there's no issues here. Format as:
\begin{align*}
    \left(\frac{3}{2}\right)! &= \Gamma \left(\frac{3}{2} + 1 \right) \\
    &= \frac{3}{4} \sqrt{\pi} \\
    \text{rest of working as is currently presented}
\end{align*}

\section{GABE12}
You've accidentally typed $V_{2}(r)$ when you meant to type $V_{3}(r)$.

\section{GABE14}
This isn't a series (sum). This is a product. Change this last sentence to ``Letting $m = \frac{n}{2}$ allows for a simpler expansion of the expression, namely:".
\par You misuse the word ``series" twice in the next paragraph.

\section{GABE15}
You can make this argument more rigorous by observing the following:
\begin{align*}
    \frac{\pi^{m}}{m!} &= \frac{\pi}{1} \times \frac{\pi}{2} \times \cdots \frac{\pi}{m-1} \times \frac{\pi}{m} \\
    &\leq \frac{\pi}{1} \times \frac{\pi}{1} \times \cdots \frac{\pi}{1} \times \frac{\pi}{m} \\ %Only the last term is unaffected
    &\to 0
\end{align*}
and also,
\begin{align*}
    0 \leq \frac{\pi^{m}}{m!}
\end{align*}
Thus, by the squeeze theorem,
\begin{align*}
    0 \leq &\frac{\pi^{m}}{m!} \leq \frac{\pi}{1} \times \frac{\pi}{1} \times \cdots \frac{\pi}{1} \times \frac{\pi}{m} \\
    \therefore \lim_{m\to\infty} 0
        \leq \lim_{m\to\infty} &\frac{\pi^m}{m!}
        \leq \lim_{m\to\infty} \frac{\pi}{1} \times \frac{\pi}{1} \times \cdots \frac{\pi}{1} \times \frac{\pi}{m} \\
    0\leq \lim_{m\to\infty} &\frac{\pi^m}{m!} \leq 0 \\
    \therefore \lim_{m\to\infty} &\frac{\pi^m}{m!} = 0
\end{align*}
This reasoning is more rigorous than ``[...] the fractions are being multiplied by increasingly smaller fractions in the series (equation I can't be bothered to type here). As a result, as $m$ increases, this series will approach $0$.", which I feel is a bit hand-wavey.

\section{GABE16}
The vertical axis should read $V_{n} (1)$.

\section{GABE17}
In reference to ``Note the dotted lines in integer intervals of dimension $n$ indicate that the fractional dimensions are undefined.", what does this mean? Is this just your graphing package dying or is something more important going on?
\par Re-read this. You explain this better at Figure \ref{fig:All volumes comparison} than you do here. Say the same thing as what you said in Figure \ref{fig:All volumes comparison}.
\par Also, I \textbf{need} this graphing package.

\section{GABE18}
In reference to ``Due to the radius of the IK-Sphere expanding to `kiss' the unit spheres centered on the vertices of the bounding $n$-cube, if the volume of the unit $n$-spheres converges to zero as dimensionality increases, that would explain why the radius reaches out to infinity to `kiss' them -- it never can.", a short reminder that the distance between the centre of the IK-Sphere and any corner of the $n$-cube gets arbitrarily large would be appreciated here. I immediately visualised a 3D case and thought ``wtf no, the sphere can get there" before slapping myself for being stupid.

\section{GABE19}
Remove the parentheses around ``$(r,n)\to\infty$" to give instead `$r,n\to\infty$'. I don't believe that the parentheses are standard notation and it appears (to me) instead like the ordered pair $(r,n)$ as opposed to indicating that you're referring to both $r$ and $n$ going to $\infty$.

\section{GABE20}
You can avoid the limit as two things go to infinity if you first provide that $r_n = \sqrt{n} -1$. It's also more technically accurate to do this first and not speak of a limit as both things go to infinity, since you're only looking at how dimensionality increases. The radius increasing is a direct consequence of the dimensionality increasing; they're not independent, so you shouldn't notate it as if they were.

\section{GABE21}
Why are you making the input of $V_n$ be $n$? You should still be taking the limit $\lim_{n\to\infty} V_n \left(r_n \right)$, not whatever $\lim_{n\to\infty} V_n \left(n \right)$ would be.

\section{GABE22}
Misuse of the word ``series".

\section{GABE23}
The following is an argument that implicitly uses the Limit Order Theorem. You don't need to reference that you're using the Limit Order Theorem (since it's basically just common sense anyway), but if you want to, go ahead.

\section{GABE24}
$\infty \times \infty = \infty$ is perfectly well-defined? Why are you saying that it's not?

\section{GABE25}
You need to also show that the smaller terms, such as $\frac{\pi (\sqrt{2m} - 1)^2}{m}$ do not approach $0$ (to avoid the undefined expression $0\times \infty$). I've already given you a proof of that on a phone picture somewhere. It approaches $2\pi$, if I remember correctly.

\section{GABE26}
Should be writing $V_{n} (r_{n} )$ instead of $V_n$.

\section{GABE27}
You've mis-labeled $V_{n} (r_n)$ as $V_{n} (n)$ again in this graph. Please check your entire document to make further changes for it because I'm not going to leave a comment about it again (because I'm lazy).

\section{GABE28}
I'm not sure if this is supposed to sat $r_n$. Please check.

\section{GABE29}
``[...] the space, [...], between unit spheres becomes infinitely large [...]" should read `[...] the diagonal space, [...], between unit spheres becomes infinitely large [...]' or something to this effect. The unit spheres do touch each other and you could say that in that sense, the space doesn't become infinitely large (because they're touching, so in that sense, the space between them is 0). Resolve that ambiguity with a word or two.

\section{GABE30}
``[...] We found (see Chapter 2: Section \ref{section:verify intersection}) that the IK-Sphere in 4 dimensions intersected at the single coordinate $(1, 1, 1, 0) [...]$ should read `[...] We found (see Chapter 2: Section \ref{section:verify intersection}) that the IK-Sphere in 4 dimensions intersected at the coordinate $(1, 1, 1, 0)$ [...]' (I removed the word ``single"). This clarifies that $(1,1,1,0)$ is not the ONLY intersection, which ``single" may imply to a half-asleep examiner.

\section{GABE31}
``[...] the amount of space between points increases somewhat exponentially [...]" Use a different adjective other than ``exponentially". ``exponentially", of course, has a very strict definition in math and you didn't show that the space between points increases in a way that satisfies the conditions necessary to be called ``exponentially".

 % REMOVE WHEN DONE
    \begin{titlepage}

    \centering
    \vfill
    
    \large{IB Extended Essay}\\
    \large{Mathematics}\\
    
    \rule{\textwidth}{1.6pt}\vspace*{-\baselineskip}\vspace*{3.2pt}
    \rule{\textwidth}{0.4pt}\\[0.2\baselineskip]
    %{\LARGE DEALING WITH MYTHS ABOUT\\[0.3\baselineskip] HIGHER DIMENSIONS}\\[0.2\baselineskip]
    
    {\LARGE ON THE PERCEPTION OF INTRODUCTORY\\[0.3\baselineskip] GEOMETRIC RESULTS\\[0.5\baselineskip] IN HIGHER DIMENSIONS}\\[0.2\baselineskip]
    
    %{\LARGE ON THE PERCEPTION OF ELEMENTARY\\[0.3\baselineskip] RESULTS IN HIGHER DIMENSIONS}\\[0.2\baselineskip]
    
    
    \rule{\textwidth}{0.4pt}\vspace*{-\baselineskip}\vspace*{3.2pt}
    \rule{\textwidth}{1.6pt}\\[\baselineskip]
    \scshape
    RQ: To what extent is it true that an inscribed sphere will expand outside of its bounding box as dimensionality increases?
    \par
    %\vspace*{2\baselineskip}
    %Word Count: 4000 \\[\baselineskip]
    %{\Large FIRST EDITOR \\ SECOND EDITOR \\ THIRD EDITOR\par}
    %{\itshape Organisation \\ Address\par}
    \vfill
    
    \small{Word Count: 4000} \\
    \vspace*{2\baselineskip}
    {\scshape November 2021} \\
  \end{titlepage}
    
\begin {flushright}
\null \vspace {\stretch {1}}
\textit{
To those who wonder of the extra-ordinary.
}
\vspace {\stretch {2}}\null

This essay was prepared using the \LaTeX{} typesetting package.

\end {flushright}



    % if you want abstract vertically centered use this vspace thingo
%\vspace*{\fill}



%{\LARGE DEALING WITH MYTHS ABOUT\\[0.3\baselineskip] HIGHER DIMENSIONS}\\[0.2\baselineskip]

% UPDATE RESEARCH QUESTION IN THE ABSTRACT
\begin{abstract}
    \begin{center}
        \rule{\textwidth}{1.6pt}\vspace*{-\baselineskip}\vspace*{3.2pt}
        \rule{\textwidth}{0.4pt}\\[0.2\baselineskip]
            What would appear to be a sphere inside of a cube in our two or three dimensions completely inverts to a cube inside of an incomparably large sphere in higher dimensions. In this essay, we examine the curious nature of an Inscribed Kissing (IK) Sphere to understand why these counter-intuitive phenomena occur in higher dimensions and add rigour and reason to Barry Cipra's IK-Sphere problem from the first volume of \textit{What's Happening in the Mathematical Sciences}. Throughout the essay, two methods of reasoning -- an algebraic versus a geometric approach -- will be used with varying degrees of certainty versus intuition gained in each. Finally, we answer the research question: \textbf{Why does the radius of an IK-Sphere tend to infinity as dimensionality increases and does it truly intersect with its bounding box in $n\geq 4$ dimensions?}.
        \rule{\textwidth}{0.4pt}\vspace*{-\baselineskip}\vspace*{4pt}
        \rule{\textwidth}{1.6pt}\\[\baselineskip]
    \end{center}
\end{abstract}



%\vspace*{\fill}


 % UPDATE RESEARCH QUESTION IN THE ABSTRACT
    \setstretch{0.5}
    \include{contents}
    \doublespacing
    \nomenclature[N]{\(\mathbb{R}\)}{Real numbers}
\nomenclature[N]{\(\mathbb{N}\)}{Natural numbers}
\nomenclature[O]{\(\mathbb{R}^n\)}{$n$-Dimensional Standard Euclidean Space}
\nomenclature[O]{$\forall$}{For All (e.g. $\forall A, B$ occurs)}
\nomenclature[O]{$\ni$}{Such That}
\nomenclature[O]{$\in$}{In (e.g. $n\in \mathbb{N}$)}
\nomenclature[L]{\(\langle x, y, z \rangle\)}{Vector (Rectangular Notation)}
\nomenclature[L]{\((x, y, z)\)}{Point}
\nomenclature[L]{\(\begin{bmatrix}a & b\\
c & d\end{bmatrix}\)}{Matrix (Bracket Notation)}

\printnomenclature
    
    \chapter{Introduction}
    \section{A Deficit of Higher Dimensional Intuition}
Our intuition was formed in three dimensions and is often misleading in higher dimensions. Consequently, we resort to unsound dimensional analogy where our observations from two or three dimensions are assumed to generalise to higher dimensions. In Volume 1 of the yearly release of \textit{What’s Happening in the Mathematical Sciences}, two such false higher dimensions conjectures proposed were presented in the chapter \textit{Disproving the Obvious in Higher Dimensions} \cite{Cipra_1993}. As an example for the reader to explore their own high-dimensional intuition, the end of the chapter featured a box titled \textit{Here's Looking at Euclid}. Within the box, without much elaboration, was the IK-Sphere Problem.

\section{The IK-Sphere Problem}
Suppose a square with a side length of 2 units in two-dimensional space has unit circles (circles with a radius of 1 unit) on each of its vertices as in Figure \ref{fig:2d_Setup_IK_Sphere}\footnote{All illustrations and graphs in this essay were made by the author}.
\begin{figure}[H]
    \centering
    \includegraphics[width=0.4\textwidth]{images/2D.png}
    \caption{\label{fig:2d_Setup_IK_Sphere}Setup for the 2D IK-Sphere problem}
\end{figure}
Now, draw a circle at the centre of the square that `kisses' the unit circles (see Figure \ref{fig:2d_IK_Sphere}). This inner circle will be named an IK-Sphere (Inscribed Kissing Sphere) for the purposes of this essay.
\begin{figure}[H]
    \centering
    \includegraphics[width=0.4\textwidth]{images/2D IK.png}
    \caption{\label{fig:2d_IK_Sphere}The 2D IK-Sphere problem.}
\end{figure}
The definition of a sphere and cube can be generalised across all dimensions so that we may extend the IK-Sphere into a higher dimensional space. We will name these the $n$-sphere and $n$-cube respectively.

\begin{definition}[$n$-Sphere]\label{def:n-sphere}
    An $n$-sphere is a generalised 3D sphere (3-sphere) to \\$n$-dimensional space, created by all points equidistant (specified by a radius) from a common centre point. For example, a circle is a 2-sphere, a line is a 1-sphere, and a point is a 0-sphere.
\end{definition}

\begin{definition}[$n$-Cube]\label{def:n-cube}
    An $n$-cube generalises the cube (3-cube) to $n$ dimensions. It is created by extending an ($n-1$)-cube in a direction perpendicular to itself (see Figure \ref{fig:how to n cube}). All edges must be of the same length.
    \begin{figure}[H]
    \centering
    \includegraphics[width=0.8\textwidth]{images/how to make n sphere.png}
    \caption{\label{fig:how to n cube}Extending an ($n-1$)-cube in a perpendicular direction to create an $n$-cube}
    \end{figure}
\end{definition}
\noindent Abiding by these definitions, we repeat the process to create an IK-Sphere in three dimensions (see Figure \ref{fig:3d_IK_Sphere}) and find that the radius of the IK-Sphere increases (see Figure \ref{fig:compare IK spheres}). 

\begin{figure}[H]
    \centering
    \includegraphics[width=0.3\textwidth]{images/3D IK.png}
    \caption{\label{fig:3d_IK_Sphere}3D IK-Sphere.}
\end{figure}

\begin{figure}[H]
    \centering
    \includegraphics[width=0.7\textwidth]{images/compare ik spheres.png}
    \caption{\label{fig:compare IK spheres}Comparing radius of 2D IK-Sphere to 3D IK-Sphere.}
\end{figure}

Without a rigorous explanation as to why, the \textit{Here's Looking at Euclid} box states that as the number of dimensions increase, the radius of the IK-Sphere increases without bound -- contrary to one's assumption that the radius will approach touching the bounding cube but never exceed it. Thus, this prompted the research question (RQ) to understand and verify the conclusions proposed about IK-Spheres: \researchquestion{}















    
    \chapter{Body: Algebraic Reasoning}
    %\epigraph{Research Question}{\begin{flushleft}\textit{Why does the radius of an IK-Sphere tend to infinity as dimensionality increases?}\end{flushleft}}
    \epigraph{\textit{Why  does  the  radius  of  an  IK-Sphere  tend  to  infinity  as  dimensionality increases and does it truly intersect with its bounding box in $n \geq 4$ dimensions?}}{Research Question}
    \noindent Algebraic reasoning aims to abstract the IK-Sphere problem and formally generalise it to the unfamiliar four and above dimensions.

\section{The Radius of the IK-Sphere as Dimension Tends to Infinity}
The radius $r$ of the IK-Sphere can be solved for algebraically and generalised by extending the Pythagorean Theorem.
\begin{lemma}[Extended Pythagorean Theorem for $n$-cube]\label{lemma:extend pythag}
    For an $n$-cube in $n$-dimensions, with a set of lines of length $a_1, a_2, a_3, ... , a_n$, each perpendicular to all others and originating from the same vertex, the length of the diagonal $c$ of the $n$-cube is given by
    \begin{equation}\label{extendedpythag}
        c^2 = \sum_{i}^{n}a_i^2
    \end{equation}
\end{lemma}
\begin{proof}

    Assuming the original two-dimensional Pythagorean Theorem is proven,
    
    \noindent 
    Base case $\left(n=3\right)$
    \begin{equation*}
        \begin{split}
            c^2&=1^2+1^2+1^2\\
            c&=\sqrt{3}
        \end{split}
    \end{equation*}
    So, the lemma holds for $n=3$.
    
    \noindent Inductive hypothesis: Suppose the theorem holds for all values of $n$ up to some $k$, $k \geq 3$.
    \begin{equation*}
        \begin{split}
            c^2=\sum_{i}^{k}a_i^2
        \end{split}
    \end{equation*}
    
    \noindent Inductive step: Let $n=k+1$. 
    \begin{equation*}
        \begin{split}
        \sum_{i}^{k+1}a^2_{i}
        \end{split}
    \end{equation*}
    Then our right side is
    \begin{equation*}
        \begin{split}
        \sum_{i}^{k}a_i^2+a_{i+1}^2=\sum_{i}^{k+1}a^2_{i}
        \end{split}
    \end{equation*}
    which is our left side. So, the theorem holds for $n=k+1$. 
    By the principle of mathematical induction, the theorem holds for all $n \in \mathbb{N}$.
\end{proof}
Let us use a diagram for a 2D IK-Sphere to generalise the IK-Sphere radius $r_n$ to $n$-dimensions. Using the Extended Pythagorean Theorem for an $n$-cube we can determine the length of a diagonal (see line $\overline{\rm AB}$ in Figure \ref{fig:2d diagonal}) of a quadrant of our 2-cube that includes a unit $n$-sphere radius and the radius of the IK-Sphere which we wish to obtain. 
\begin{figure}[h]
    \centering
    \includegraphics[width=0.4\textwidth]{images/diagonal 2d.png}
    \caption{\label{fig:2d diagonal}Unit 2-cube diagonal length determined using Extended Pythagorean Theorem}
\end{figure}

\noindent Hence, we can subtract the unit circle radius to obtain the radius of the two-dimensional IK-Sphere: $$r_2 = \sqrt{1^2+1^2}-1.$$  %Then, dividing by 2, we get the radius of an IK-Sphere in two-dimensions: $$r_2=\frac{\sqrt{1^2+1^2}-2}{2}.$$ 

Similarly, for a three-dimensional IK-Sphere, the length of the diagonal is the square root of the sum of three sides of of an octant of the $n$-cube at a vertex (see Figure \ref{fig:3d diagonal}). From there, the process is the same and we subtract the unit sphere. Thus, we have an inscribed sphere radius $r_3$ of: 

\begin{equation*}
    r_3=\sqrt{1^2+1^2+1^2}-1.
\end{equation*}

\begin{figure}[H]
    \centering
    \includegraphics[width=0.6\textwidth]{images/3d diagonal.png}
    \caption{\label{fig:3d diagonal}Unit 3-cube diagonal length determined using Extended Pythagorean Theorem}
\end{figure}

\begin{corollary}[To the Extended Pythagorean Theorem for $n$-Cubes] The diagonal $c$ of a unit $n$-cube is given by
\begin{equation} \label{ncube diagonal}
    c = \sqrt{n}
\end{equation}
\end{corollary}
\begin{proof}
    Substituting $a=1$ into the extended Pythagorean Theorem (see Equation \ref{extendedpythag}), we have
    \begin{align*}
        c^2 &= \sum_{i}^{n}1_i^2\\
        &=n\\
        \therefore c &= \sqrt{n}.
    \end{align*}
\end{proof}

If the diagonal of a unit $n$-cube is given by $\sqrt{n}$ (see Equation \ref{ncube diagonal}), the IK-Sphere radius can be generalised across all $n$-dimensions by subtracting the unit sphere radius as follows:
\begin{equation}\label{radius of ik sphere}
    r_n = \sqrt{n}-1
\end{equation}

We may now proceed to the final step of proving that the radius of an IK-Sphere increases without bound.

\begin{theorem}
As the dimension $n$ tends to infinity, the radius $r_n$ of the inscribed sphere also tends to infinity.
\end{theorem}
\begin{proof}
Take the limit as $n$ approaches infinity of the expression for $r_n$ in terms of $n$ (See Equation \ref{radius of ik sphere}) and apply the difference law:
\begin{align*}
    \lim_{n\to\infty} \sqrt{n}-1 &= \lim_{n\to\infty} \sqrt{n} - \lim_{n\to\infty} \sqrt{1}\\
    &=\infty - 1\\
    &=\infty
\end{align*}
$\therefore$ the radius of an IK-Sphere diverges to infinity as dimension $n$ increases.\\
\end{proof}

Graphically (see Figure \ref{fig:radius increases graph}), we may observe the same result with less rigour. The domain has been restricted to $n\geq3$ as $n=3$ was the base case for the Extended Pythagorean Theorem for an $n$-Cube (see Lemma \ref{lemma:extend pythag}). Below $n=1$ we obtain negative values for $r_n$ which can accordingly be considered undefined region. Investigation into fractional dimensions could reveal a reason, however, that is currently beyond the scope of this essay.
%Here begins the 2D plot
\begin{figure}[H]
    \centering
    \begin{tikzpicture}
        \begin{axis}[
            axis lines = left,
            xlabel = \(n\),
            ylabel = {\(r_n\)},
        ]
        %Below the red parabola is defined
        \addplot [
            domain=3:100, 
            samples=100, 
            color=red,
        ]
        {sqrt(x)-1};
        \addlegendentry{\(\sqrt{n}-1\)}
        %Here the blue parabola is defined
        \end{axis}
    \end{tikzpicture}
    \caption{Caption}
    \label{fig:radius increases graph}
\end{figure}
%Here ends the 2D plot






\section{Verification of the IK-Sphere at Nine Dimensions}
The radius of an IK-Sphere has now been proven to increase without bound. As such, one may infer that the IK-Sphere bursts outside of its bounding box as dimensionality increases, implying that there exist intersection points between the IK-Sphere and its `bounding' $n$-cube at higher dimensions. In fact, upon inspection, one may observe that at $n=4$ dimensions, $$r_4=\sqrt{4}-1=1.$$ The IK-Sphere fits perfectly within its bounding 4-cube with side length 2 units (!). Here, intuition takes hold once again. One may assume there exists a solvable intersection point on a single face of the 4-cube with the IK-Sphere. However, another possibility arises in which higher dimensional intersections may operate differently and the IK-Sphere may still exist within its bounding $n$-cube as we originally assumed. We can verify the existence, or lack thereof, of an intersection point to decide which of these possibilities is true in higher dimensions.

From the definition of an $n$-sphere (see Definition \ref{def:n-sphere}) the following equation can be constructed:
\begin{equation}\label{eq:unit 4-sphere}
    x_1^2+y_2^2+z_3^2+w_4^2=1
\end{equation}
where $(x, y, z, w)$ is a point on the 4-sphere.

However, this does not help our high dimensional intuition. As such, geometric reasoning is required.


    
    \chapter{Body: Geometric Reasoning}
    \epigraph{\textit{Why  does  the  radius  of  an  IK-Sphere  tend  to  infinity  as  dimensionality increases and does it truly intersect with its bounding box in $n \geq 4$ dimensions?}}{Research Question}
    A geometric reasoning can aid in developing a better intuition for higher dimensional phenomena. This chapter aims to reason for why the IK-Sphere radius diverges in terms of space consumption of the IK-Sphere and the unit $n$-spheres it `kisses'.

\begin{definition}[Volume]
    Here, volume has been generalised across $n$ dimensions as the number of unit $n$-cubes of space a shape occupies. For example, `area' is volume in two-dimensional space, as is `volume' in three-dimensional space. 
\end{definition}

\section{The Volume of the $n$-Sphere}

\section{Implications of the Higher Dimensional Unit Sphere on the IK-Sphere}
    
    \chapter{Conclusion}
    \section{Conclusion}
In conclusion, to answer the research question:
\researchquestion{}

There were two parts to the question. Firstly, we answer why the radius of an IK-Sphere tends to infinity as dimensionality increases:
The algebraic reason would be that when the limit of the expression for the radius $r$ of an IK-Sphere in terms of dimension $n$ is evaluated as $n$ approaches infinity, the radius tends to infinity as follows (see Theorem \ref{theorem:radius of IK-Sphere}):
\begin{align*} %GABE28: I'm not sure if this is supposed to sat $r_n$. Please check.
    r = \sqrt{n}-1\\
    \lim_{n \to \infty}\sqrt{n}-1 = \infty
\end{align*}

%GABE29: ``[...] the space, [...], between unit spheres becomes infinitely large [...]" should read `[...] the diagonal space, [...], between unit spheres becomes infinitely large [...]' or something to this effect. The unit spheres do touch each other and you could say that in that sense, the space doesn't become infinitely large (because they're touching, so in that sense, the space between them is 0). Resolve that ambiguity with a word or two.
However, such abstraction is not very useful in understanding higher dimensions. As an aid to one's higher dimensional intuition, the geometric reason is that as the dimension increases, the space, determined by the volume of the unit $n$-spheres the IK-Sphere `kisses', between unit spheres becomes infinitely large because the unit $n$-spheres' volumes approach zero (see Lemma \ref{lemma:unit sphere volume tends to zero}). Additionally, the volume of the bounding $n$-cube on which those spheres are situated also tends to infinity. As a result, the radius of the IK-Sphere expands to fill the space and its volume also tends to infinity (see Theorem \ref{theorem:volume of IK-Sphere}) in an attempt to `kiss' its bounding unit spheres. 

%GABE30: ``[...] We found (see Chapter 2: Section \ref{section:verify intersection}) that the IK-Sphere in 4 dimensions intersected at the single coordinate $(1, 1, 1, 0) [...]$ should read `[...] We found (see Chapter 2: Section \ref{section:verify intersection}) that the IK-Sphere in 4 dimensions intersected at the coordinate $(1, 1, 1, 0)$ [...]' (I removed the word ``single"). This clarifies that $(1,1,1,0)$ is not the ONLY intersection, which ``single" may imply to a half-asleep examiner.
The second part of the question was whether the IK-Sphere truly intersected with its `bounding' box in $n>4$ dimensions or if it was merely higher dimensional phenomena making it seem so. We found (see Chapter 2: Section \ref{section:verify intersection}) that the IK-Sphere in 4 dimensions intersected at the single coordinate $(1, 1, 1, 0)$, directly in the center of the face of its bounding cube's $w=0$ face, just as expected. Thus, we also conclude that there is truth to an IK-Sphere intersecting with its bounding $n$-cube as dimensionality increases past $n=4$.

\section{Implications and Further Investigation} 
%GABE31: ``[...] the amount of space between points increases somewhat exponentially [...]" Use a different adjective other than ``exponentially". ``exponentially", of course, has a very strict definition in math and you didn't show that the space between points increases in a way that satisfies the conditions necessary to be called ``exponentially". Nice choice of application, by the way.
The conclusion has implications relating to what is known as the curse of dimensionality for high dimensional data sets in machine learning \cite{raviv2020perm2vec}. We found that as you increase dimensions, that the amount of space between points increases somewhat exponentially (see Figure \ref{fig:IK sphere volume to infinity graph} and `Bounding $n$-Cube Volume' in Figure \ref{fig:All volumes comparison}). As a result, when data has too many variables, the machine learning algorithm has trouble pinpointing attributes and patterns because the data is so vastly spread out. One may investigate possible dimensional elimination techniques to mitigate this issue. Additionally, the conclusion has implications for error correcting codes. For example, when a CD or DVD has a scratch on it, changing the encoded data, an error correcting code is used to fill in and `repair' the missing data. If each coordinate in high dimensional space is a piece of data, the space is separated into as many spheres as possible and the data read from the disk is corrected to the coordinate of the center of the sphere it exists within. Thus, the closer together we can get the spheres, the more accurate data corrections we can achieve. However, we learned that the space between spheres can become increasingly larger as dimension increases which means that as dimensions increase, even more spheres can be packed around each other. Until now, this sphere packing problem has only been proven up to eight dimensions and in 24 dimensions \cite{klarreich2016sphere}. With our newfound intuition of higher dimensions, could we prove the remaining dimensions? It is unlikely. However, such an investigation could reveal more phenomena to add to our bank of higher dimensional intuition.
    
    \printbibliography
    %\printbibliography[type=inbook, title={Books}]
    
    \appendix
    \chapter{Appendix A}
    \section{Original IK-Spheres in \textit{What's Happening in the Mathematical Sciences}}
\centering
\includegraphics[width=0.6\textwidth]{images/spheres.png}
\end{document}
